%%%%%%%%%%%%%%%%%%%%%%%%%%%%%%%%%%%%%%%%%
% Thin Sectioned Essay
% LaTeX Template
% Version 1.0 (3/8/13)
%
% This template has been downloaded from:
% http://www.LaTeXTemplates.com
%
% Original Author:
% Nicolas Diaz (nsdiaz@uc.cl) with extensive modifications by:
% Vel (vel@latextemplates.com)
%
% License:
% CC BY-NC-SA 3.0 (http://creativecommons.org/licenses/by-nc-sa/3.0/)
%
%%%%%%%%%%%%%%%%%%%%%%%%%%%%%%%%%%%%%%%%%

%----------------------------------------------------------------------------------------
%	PACKAGES AND OTHER DOCUMENT CONFIGURATIONS
%----------------------------------------------------------------------------------------

\documentclass[a4paper, 11pt]{article} % Font size (can be 10pt, 11pt or 12pt) and paper size (remove a4paper for US letter paper)

\usepackage[usenames, dvipsnames]{color}
\usepackage[protrusion=true,expansion=true]{microtype} % Better typography
\usepackage{subfigure}
\usepackage{graphicx} % Required for including pictures
\usepackage{wrapfig} % Allows in-line images

\usepackage{mathpazo} % Use the Palatino font
\usepackage[T1]{fontenc} % Required for accented characters



\usepackage{url} % para que aparezcan las url de las paginas web. En el archivo bib hay que poner howpublished = "\url{URL del recurso}"
\usepackage[nottoc,numbib]{tocbibind} % para que la bibliografia aparezca en la tabla de contenidos (toc) numerada adecuadamente (numbib)
\usepackage{listings} % para introducir codigo bonito
\usepackage{textcomp} % hace falta para poner apostrofe ' -.-''''''''
\usepackage{gensymb} % para que \degree introduzca el simbolo 
\usepackage[hidelinks]{hyperref} %clickable index
\hypersetup{
    colorlinks=false, %set true if you want colored links
    linktoc=all,     %set to all if you want both sections and subsections linked
%    linkcolor=blue,  %choose some color if you want links to stand out
}

\usepackage[export]{adjustbox} % Para generar bordes negros en las figuras

%\definecolor{black}{rgb}{0,0,0} % Example color definition, the color can be used with the \color{name} command
%\definecolor{red}{rgb}{1,0,0} % Example color definition, the color can be used with the \color{name} command

\linespread{1.05} % Change line spacing here, Palatino benefits from a slight increase by default

\makeatletter
\renewcommand\@biblabel[1]{\textbf{#1.}} % Change the square brackets for each bibliography item from '[1]' to '1.'
\renewcommand{\@listI}{\itemsep=0pt} % Reduce the space between items in the itemize and enumerate environments and the bibliography

\renewcommand{\maketitle}{ % Customize the title - do not edit title and author name here, see the TITLE block below
\begin{flushright} % Right align
{\LARGE\@title} % Increase the font size of the title

\vspace{50pt} % Some vertical space between the title and author name

{\large\@author} % Author name
\\\@date % Date

\vspace{40pt} % Some vertical space between the author block and abstract
\end{flushright}
}

\usepackage[english]{babel}
 
\setlength{\parindent}{2em}
\setlength{\parskip}{1em}


%----------------------------------------------------------------------------------------
%	TITLE
%----------------------------------------------------------------------------------------

\title{\textbf{Structural Bioinformatics Final Exam}\\  % Title
\vspace{10pt}Protein Part} % Subtitle

\author{\textsc{Antonio Ortega Jim\'enez} % Author
\\{\textit{University of Copenhagen}}} % Institution

\date{\today} % Date

%----------------------------------------------------------------------------------------

\begin{document}

\maketitle % Print the title section
\thispagestyle{empty}

\vspace{-3.7cm}
\begin{figure}[!h]
\includegraphics[scale=0.35]{figures/ku_logo}
\end{figure}

\vspace{30pt}

\begin{center}
\Huge{\textbf{RMSD analysis on identical n-mers in the top100H database}}
\end{center}

%\begin{figure}[!h]
%\begin{center}
%\includegraphics[scale=0.28,frame]{figures/portada_1}% tight frame
%\end{center}
%\end{figure}




\newpage
%----------------------------------------------------------------------------------------
%	INDICE
%----------------------------------------------------------------------------------------

\setcounter{figure}{0}
%\renewcommand\thefigure{\arabic{figure}}
%\renewcommand{\figurename}{\textbf{Figura}}
%
%\setcounter{table}{0}
%\renewcommand\thetable{\arabic{table}}
%\renewcommand{\tablename}{\textbf{Tabla}}

\thispagestyle{empty}
\tableofcontents
\thispagestyle{empty}

\newpage




%----------------------------------------------------------------------------------------
%	ABSTRACT AND KEYWORDS
%----------------------------------------------------------------------------------------


%\renewcommand{\abstractname}{Resumen} % Uncomment to change the name of the abstract to something else


%\begin{abstract}
%%hay que repasarlo
%Se desarrollaron diversos experimentos con el fin de profundizar en los m\'etodos de producci\'on y control de calidad en la industria alimentaria.
%El an\'alisis del gluten en los alimentos es de inter\'es debido a la problem\'atica de los cel\'iacos. Como prueba de concepto se midi\'o de forma semicuantitativa la presencia de gluten en galletas del supermercado El Jam\'on. Uno de los principales alicientes de los zumos de fruta es la riqueza en vitamina C, un nutriente esencial. Por ello, se valor\'o la concentraci\'on de ascorbato (vitamina C) en zumos de la marca Juver. Los yogures son uno de los productos l\'acteos m\'as conocidos, debido al aporte de probi\'oticos que supone. A partir de in\'oculos de diferentes yogures comerciales elaboramos m\'as yogur y durante el proceso, se sigui\'o el descenso del pH de la leche, un indicador de la evouci\'on de la fermentaci\'on. Tratamos de aislar los microorganismos responsables con medios selectivos est\'andar. Por \'ultimo, se registr\'o  la fermentaci\'on de la harina comercial debida a diferentes cepas de levadura.
%\end{abstract}

\vspace{30pt} % Some vertical space between the abstract and first section

%----------------------------------------------------------------------------------------
%	ESSAY BODY
%----------------------------------------------------------------------------------------

\setcounter{page}{1}
\section{Introduction}

The study of protein structure is a hot topic on several fields like medicine and the biotech industry. For that matter, Bioinformatics methods that predict and analyze structure information are constantly being developed by the scientific community. Moreover, since research groups all around the world publish their results on open databases this methods can be easily be applied to real data. One of such databases is the Protein Data Bank, maintained by an international consortium. In other words, bioinformaticians have both open data and free software to try out new ideas.


Do we expect structure similarity in identical sequences of n residues? In this report, a simple \textit{in silico} analysis of protein structure data was run to try to give an answer to this question. The Biopython module, contained in the Python programming language, together with the statistical language R, were used for this purpose.


\section{Methods}

\subsection{Build of the top100H}

The top100H database is available in this link
\color{blue}{
\href{http://kinemage.biochem.duke.edu/databases/top100.php}{http://kinemage.biochem.duke.edu/databases/top100.php}
}

\section{Results}

\section{Discussion}


%Se emple el dataset de secuenciacin masiva proveniente de ChIP-seq y RNA-seq disponible en la base de datos GEO bajo el identificador GSE43286. Estos datos pertenecen al trabajo de Pfeiffer A. \textit{et al} \cite{Pfeiffer2014}.
%
%\begin{itemize}
%
%\item 8 muestras de RNA-seq organizadas en 4 condiciones con 2 rplicas cada una.
%
% \begin{enumerate}
% \item WT: todos los PIF activos
% \item pif1: solo PIF1 est activo
% \item pif4: solo PIF4 est activo
% \item quartet: todos los PIF estn anulados
% \end{enumerate}
%
%\item 9 muestras de ChIP-seq organizadas en 3 condiciones con 3 rplicas cada una.
% \begin{enumerate}
% \item input: no se selecciona un cistroma especfico
% \item chip-pif1: se selecciona el cistroma de pif1
% \item chip-pif4: se selecciona el cistroma de pif4
% \end{enumerate}
%
%\end{itemize}
%
%\subsubsection{Procesamiento de muestras de RNA-seq y ChIP-seq}
%El anlisis de RNA-seq se hizo con el protocolo Tuxedo (TopHat, Cufflinks, Cuffmerge, Cuffdiff y CummeRbund) seleccionando genes con un valor absoluto de fold-change > 2 y un p-valor < 0.05. Para analizar muestras de ChIP-seq se utilizaron los programas MACS y PeakAnnotator.
%
%\subsubsection{Anlisis de picos de ChIP-seq}
%Para analizar los datos masivos provenientes de plataformas de ChIP-seq se utilizaron diversos paquetes alojados en Bioconductor, un repositorio de software de anlisis de datos biolgicos en el lenguaje de programacin R.
%
%Los paquetes GenomicRanges y IRanges proporcionan el tipo de dato necesario para analizar el cistroma. Los datos de secuenciacin se pueden interpretar como intervalos de una secuencia (el genoma), a los cuales se les puede asociar informacin adicional. As, estos paquetes proporcionan los GRanges, una estructura de datos diseada para almacenar intervalos de secuencias y aadirles metadatos de forma eficiente y sencilla.
%
%El paquete ChIPseeker \cite{chipseeker} implementa diversos mtodos de anlisis que simplifican la integracin de los ficheros de picos de ms de un factor de transcripcin. Facilita la generacin de grficos y la elaboracin de anlisis estadsticos, adems del acceso a otros \textit{datasets}. Para ello se basa en otros paquetes como clusterProfiler, UpSetR y ggplot2. Tambin hace uso de bases de datos de transcritos del formato txdb, accesibles desde Bioconductor.
%
%\subsubsection{Integracin con RNA-seq}
%Los picos de ChIP-seq se filtraron con las regiones promotoras de genes expresados diferencialmente entre condiciones de inters. Las 4 condiciones exploradas permiten caracterizar genes que no solo respondan a la unin de uno de los PIFs a su promotor, sino que tambin a la unin simultnea de ambos. Es decir, nos permiten detectar fenmenos de expresin diferencial que pudieran estar causados por una interaccin entre los 2 FTs unidos al mismo promotor. Esta interaccin podra ser tanto indirecta (contacto fsico) como indirecta.
%
%Esta coincidencia entre a) DEGs por accin de los PIFs y b) solapamiento significativo entre promotores de DEGs y los picos de ChIP-seq, es la base del mtodo implementado para predecir \textit{in silico} una posible interaccin.
%
%\subsection{Resultados y discusin}
%
%\subsubsection{El perfil de los picos de PIF1 y PIF4 es similar}
%
%Los estudios exploratorios indican que las dianas de ambos protenas se encuentran solo unos cientos de bases \textit{upstream} de los TSS (figura \ref{fig:profile} a), lo que confirma sus propiedades de factores de transcripcin. Adems, el patrn formado en la unin a las regiones promotoras (figura \ref{fig:profile} b) es muy parecido, lo que proporciona evidencias de una posible actuacin en comn.
%
%\begin{figure}[!h]
%\subfigure[Frecuencia de picos a lo largo de promotores]{\includegraphics[scale=0.30]{figures/count_profile.png}}
%\subfigure[Mapas de calor]{\includegraphics[scale=0.40]{figures/heatmap_profile.png}}
%\caption{Perfil de unin de PIF1 y PIF4 a las regiones promotoras. El sombreado muestra el intervalo de confianza al 95\%, determinado por bootstrap. El heatmap genera una "huella" que caracteriza la unin de ambos FTs a sus promotores diana.}
%\label{fig:profile}
%\end{figure}
%
%Desde el punto de vista funcional, tampoco se revelan grandes diferencias, teniendo ambos FTs una gran preferencia por regiones promotoras a menos de 1 kb del TSS (figura \ref{fig:anotacion}).
%
%\begin{figure}[!h]
%\centering
%\includegraphics[scale=0.30]{figures/annotation_bars}
%\caption{Anotacin funcional de las dianas de PIF1 y PIF4. En la mayor parte de los casos las dianas se encuentran en las regiones promotoras.}
%\label{fig:anotacion}
%\end{figure}
%
%\subsubsection{El anlisis estadstico indica que ambos FTs comparten dianas de forma significativa}
%
%Para concluir el anlisis exploratorio, implementamos un contraste de hiptesis que evala la significancia del solapamiento entre las dianas de ambos.
%
%$H_0$: Los ficheros de picos no solapan de manera significativa, todo solapamiento hallado es por puro azar
%
%$H_1$: El solapamiento es muy fuerte y probablemente no es aleatorio, sino que es originado por una causa subyacente.
%
%\begin{table}[!h]
%\centering
%\caption{Anlisis estadstico del solapamiento entre los ficheros de picos de PIF1 y PIF4}
%\label{table:peak-overlap-table}
%\begin{tabular}{c c c c c c}
%	Query & Target & Query length & Target length & N Overlap & p-value \\ \hline
%	PIF1  & PIF4   & 2539         & 1806          & 1194      & 0.0005
%\end{tabular}
%\end{table}
%
%Por tanto, podemos descartar la hiptesis nula con alto nivel de significancia. Los dos factores actan en las mismas dianas.
%
%\subsubsection{Se registraron 42 genes activados a lo largo de los 4 contrastes implementados}
%
%Los contrastes implementados generaron listas de genes activados con la unin de PIF1 o PIF4 y tambin con la unin simultnea de ambos. Los 42 genes hallados en comn en las 4 listas se consideraron candidatos para filtrar sus promotores con los picos procedentes de ChIP-seq.
%
%\begin{figure}[!h]
%\centering
%\label{fig:venn}
%\includegraphics[scale=0.15]{figures/progressive_activation}
%\caption{Diagrama de Venn de las cuatro condiciones exploradas. 42 genes que suponen el 2.3\% del total se observan en el cruce central \cite{venny}.}
%\end{figure}
%
%\subsubsection{3 de ellos solapan sus regiones promotoras con los picos de los PIFs}
%
%Finalmente, 3 de estos genes resultaron solapar con los picos de unin de los factores de transcripcin PIF1 y PIF4. Estos genes son:
%
%\begin{itemize}
%\item AT1G02340 (\textit{hfr1})
%\item AT1G77330 (ACC oxidasa)
%\item AT5G51210 (\textit{ol3})
%\end{itemize}
%
%\textit{hfr1} es, sorprendentemente, un gen fotomorfognico, del cual se sabe que interacciona con PIF4 impidiendo su unin al DNA. Por tanto, existe una relacin compleja entre HFR1 y PIF4 que podra constituir algn tipo de motivo de red. El software IGV \cite{IGV} nos permite visualizar el entorno genmico de \textit{hfr1} (figura \ref{fig:hfr1_igv}).  Por otro lado, la ACC oxidasa es una de las enzimas clave, junto a la ACC sintasa, en la ruta de biosntesis de etileno. Esta hormona vegetal participa en el crecimiento vegetal, lo que apoya el papel regulador del crecimiento de los PIFs. Ms estudios seran necesarios para entender esa relacin. Finalmente \textit{ol3} es un gen que estabiliza los oleosomas de la semilla. \cite{uniprot}
%
%\begin{figure}[!h]
%\begin{center}
%\includegraphics[scale=0.28,frame]{figures/hfr1_intervalos_2}% tight frame
%\end{center}
%\caption[Entorno genmico de HFR1]{Encontramos 2 picos de cada FT muy solapantes y menos de 2000 pbs aguas arriba del sitio de inicio de la transcripcin (TSS). Adems, la expresin de este gen aumenta conforme aumenta el nmero de FTs de la familia PIF activos presentes en el organismo (\textit{tracks} WT, PIF1on y quartetmutant). Este gen demuestra la accin aditiva de los PIF, y da pie a la hiptesis del complejo.}
%\label{fig:hfr1_igv}
%\end{figure}
%
%
%El anlisis estadstico de este solapamiento tambin permite descartar la hiptesis del azar y efectivamente, habra un solapamiento significativo entre ambos conjuntos de intervalos (tabla \ref{table:peak-deg-overlap-table}). Este resultado apoya la hiptesis de una interaccin directa PIF1-PIF4.
%
%Las datos procesados e interpretados certifican que PIF1 y PIF4 comparten ampliamente sus dianas y tienen un efecto aditivo en la regulacin gnica. La magnitud de este solapamiento es tal que la existencia de interaccin entre ambos no se puede descartar.
%
%La realizacin de otros estudios de prediccin \textit{in silico} como \textit{docking} molecular (biologa estructural) o bsqueda de motivos enriquecidos (estadstica) aportara ms datos complementarios. Una vez se ha evaluado la posibilidad de la interaccin, debe validarse en el plano experimental, con tcnicas como el doble hbrido en levadura o la resonancia magntica nuclear.
%
%\begin{table}[!h]
%\centering
%\caption{Anlisis estadstico del solapamiento entre promotores de genes expresados de forma diferencial y picos de ChIP-seq}
%\label{table:peak-deg-overlap-table}
%\begin{tabular}{c c c c c c}
%	  Query    & Target & Query length & Target length & N Overlap & p-value \\ \hline
%	Promotores & Picos  & 36           & 3151          & 6         & 0.016
%\end{tabular}
%\end{table}
%
%\section{Continuacin del protocolo de transcriptmica \textit{de novo}. Anlisis de expresin diferencial en 2 muestras de \textit{Xenopus tropicalis}.}
%
%\subsection{Antecedentes}
%
%El desarrollo embrionario es un proceso biolgico muy complejo en el que participan una inmensa cantidad de factores, molculas y estmulos externos que definen y estructuran el cuerpo de los organismos multicelulares como los humanos. Su estudio tiene gran importancia para comprender el fundamento molecular de muchas enfermedades genticas y ayudar a encontrar una cura o al menos un tratamiento para los pacientes. Por otro lado, al ser un aspecto clave de la vida de los seres vivos, su estudio promete devolver un conocimiento poderoso que podra revolucionar la evolucin tal y como la conocemos.
%
%\begin{figure}[!h]
%\begin{center}
%\includegraphics[scale=0.8,frame]{figures/dev-stages}
%\end{center}
%\caption{Estadios del desarrollo de \textit{X. tropicalis} explorados en el trabajo publicado \cite{Tan2013}. En el presente trabajo se analiz una muestra de cada fase marcada.}
%\label{fig:stages}
%\end{figure}
%
%En la tarea de transcriptmica \textit{de novo} realizamos el ensamblaje del transcriptoma de 2 muestras de huevos de \textit{Xenopus tropicalis}, un organismo modelo para el desarrollo, en 2 fases larvarias distintas (figura \ref{fig:stages}). Se control la calidad del ensamblaje de Trinity y se detectaron los \textit{clusters} de ortlogos presentes en las muestras, pero rest realizar una anlisis de expresin diferencial de los transcritos. De esta manera, se podra determinar cules son aquellos transcritos que estn expresados de forma diferencial e intentar trazarlos a los genes que los originan y as trazar redes de coexpresin gnica y patrones de expresin que caractericen el desarrollo en \textit{X. tropicalis}. Siendo este un organismo vertebrado, esperamos que exista gran nmero de ortlogos con \textit{Homo sapiens} para poder correlacionar este conocimiento a nuestra especie.
%
%\subsection{Materiales y mtodos}
%
%\subsubsection{Datos utilizados}
%
%Los datos utilizados en esta prueba de concepto corresponden a los aportados en el trabajo de Tan. M \textit{et al} \cite{Tan2013}, un estudio del transcriptoma de \textit{X. tropicalis} a lo largo de varias etapas del desarrollo embrionario, desde la puesta de huevos hasta el estado de renacuajo. En concreto, se hizo uso de las muestras GSM919938 y GSM919961 (figura \ref{fig:stages}) recogidas en el dataset con el accession number GSE37452 en la base de datos GEO del NCBI. Se utilizaron el mnimo nmero de muestras y condiciones (sin rplicas) para minimizar el gasto computacional y facilitar esta prueba de concepto.
%
%\subsubsection{Anlisis matemtico-computacional: edgeR}
%
%El genoma de este organismo ha sido secuenciado, pero an se encuentra incompleto, de forma que un anlisis \textit{de novo} aporta informacin que se perdera con un alineamiento a dicha referencia. Por ejemplo, todos aquellos transcritos que sean codificados por \textit{locus} que no estuviesen dentro del ensamblado pblico se ignoraran.
%
%El software utilizado para realizar el anlisis es el propuesto para el pipeline de anlisis basado en Trinity \cite{Trinity}, que hace uso del paquete de Bioconductor edgeR \cite{edgeR}. Bioconductor es un repositorio de paquetes para el lenguaje de programacin R, especializado en el anlisis de datos de origen biolgico.
%
%El protocolo arranc con la matriz de cuantificacin de transcritos que gener el software RSEM a partir de las lecturas, en formato .fastq, y el ensamblado Trinity.fasta. Se realiz el contraste entre las dos muestras y se determinaron diversos parmetros estadsticos que describen el comportamiento de cada transcrito.
%
%A continuacin, se seleccionaron aquellos que cumplieron el criterio de expresin diferencial, fijado en un fold change > 2 y un p-valor < 0.001. El p-valor se estima aplicando mtodos de mxima verosimilitud condicional ajustados a cuantiles (qCML) y el test exacto de Fisher.
%
%Los transcritos con mayor carcter diferencial se corrieron contra BLAST para determinar por homologa de secuencia su identidad y as adquirir una interpretacin biolgica de los datos.
%
%Finalmente, se realiz una extraccin de \textit{clusters} de genes que tuvieron un patrn de expresin similar a lo largo de las condiciones. Al solo haber 2 condiciones, solo habrn 2 patrones posibles, de ah que en este caso, esa informacin no sea de relevancia.
%
%\subsection{Resultados y discusin}
%
%\subsubsection{Miles de genes cambian significativamente su expresin desde el estado de 2 clulas hasta la organognesis}
%Se detectaron 10603 transcritos expresados diferencialmente. De ellos, 2981 tuvieron una expresin significativamente mayor en la fase de estado de 2 clulas, mientras que el nmero hallado en la fase embrionaria fue de 7622. Esta gran diferencia se puede entender a la luz de la mucha mayor complejidad de procesos y regulacin gentica presentes en un embrin en desarrollo.
%
%La tasa de expresin diferencial se situ en el 13.67 \% del total de transcritos ensamblados por Trinity.
%
%
%\subsubsection{Anotacin funcional de genes expresados diferencialmente}
%La representacin grfica de la seleccin de genes expresados de forma diferencial (DEGs) es el volcano plot (figura \ref{fig:volcano}).
%
%\begin{figure}[!h]
%\begin{center}
%\includegraphics[scale=0.57]{figures/volcano}% tight frame
%\end{center}
%\caption[Volcano plot]{Volcano plot de la expresin diferencial de los transcritos de Trinity en el contraste 2 clulas vs Organognesis. Cada punto representa un transcrito (contig de Trinity). El eje Y representa la significancia estadstica, mientras que el eje X representa el fold change, en ambos sentidos. \textcolor{red}{Rojo:} expresado diferencialmente. \textbf{Negro:} no expresado diferencialmente.}
%\label{fig:volcano}
%\end{figure}
%
%Se observ el clsico patrn de dispersin. La mayora de los transcritos se concentraron en la zona no diferencial de bajo fold change y p valor no significativo, mientras que una cierta cantidad se situ en la zona diferencial. Adems, se visualiza que, a mayor fold change, mayor significancia estadstica.
%
%
%Algunos de los transcritos destacables son los determinados por BLAST como correspondientes a \textit{foxh1}, \textit{actc1} y \textit{col2a1}.
%
%\begin{enumerate}
%\item \textit{foxh1}: Se trata de un factor de transcripcin que responde a induccin por TGF-$\beta$ formando el complejo ARF1 con las protenas smad2 y smad4 \cite{uniprot}. Por tanto, se encuentra dentro de una de las rutas metablicas que KAAS detect como activas en el anlisis de ortlogos realizado en otro trabajo anterior. Se expres con mayor intensidad en la fase de 2 clulas, lo que indica su papel clave en el diseo ms bsico del cuerpo.
%
%\item \textit{col2a1}: Codifica el colgeno de tipo II, especfico de tejido cartilaginoso. Este tejido es fundamental en el desarrollo del embrin \cite{uniprot}, pues es el responsable de aportar el soporte estructural en el que se empiezan a formar los rganos en la organognesis. As, sera un gen clave del proceso, pues su pico de expresin se produjo en esta fase. La acondroplasia, un sndrome de origen gentico que genera enanismo, se debe a la formacin de tejido seo temprana durante el desarrollo, lo que frena el crecimiento lineal del individuo. La investigacin en este tipo de genes del desarrollo podra generar conocimiento til en el  tratamiento o la cura de esta y otras enfermedades.
%
%\item \textit{actc1}: Codifica actina, una de las protenas, junto a la miosina, responsables de generar las fibras musculares. Tambin se expres con mayor intensidad en la fase de organognesis.
%\end{enumerate}
%
%En conclusin, este tipo de anlisis permite caracterizar los picos de expresin de diversos transcritos (genes) a lo largo de las condiciones exploradas, lo que permite atribuirles una funcin biolgica determinada y permite orientar la validacin experimental en el laboratorio.
%
%\subsubsection{El estadstico E90N50 refina el N50}
%
%La figura \ref{fig:volcano} nos muestra una cantidad de transcritos significativa que se agrupan en torno a una banda en la periferia del grfico. El anlisis de la matriz de cuantificacin de transcritos producido por RSEM revel que el factor diferenciador de estos transcritos es su completa ausencia en una de las 2 condiciones ensayadas. Es decir, RSEM no pudo procesar los archivos de lectura para cuantificar todos los transcritos de Trinity en todas las condiciones.
%
%El que un transcrito fuese cuantificado en una condicin pero no en la otra lo convierte automticamente en DE segn nuestro criterio de seleccin. En cualquier caso, el hecho de que no pudiese ser cuantificado en una de las condiciones refleja su baja expresin, por lo que no sera errneo considerarlo DE. En resumen, estos transcritos no alcanzaron el umbral mnimo de expresin en una de las condiciones, pero a pesar de tener una distribucin anormal, este fenmeno no da pie a interpretaciones errneas.
%
%\begin{figure}[!h]
%\begin{center}
%\includegraphics[scale=0.40]{figures/longitud_diferencial}
%\end{center}
%\caption[Distribucin de la longitud de los transcritos]{Distribucin de la longitud de los transcritos. Conforme aumentamos el nmero de muestras en las que se cuantifican los transcritos, la distribucin de longitudes se hace ms grande.}
%\label{fig:cortos}
%\end{figure}
%
%
%Por otro lado, es de esperar que al igual que hubo transcritos que no se cuantificaron en una de las condiciones, tambin los hubiese que no se cuantificaron en ninguna de las dos. Efectivamente, 17107 tuvieron una cuantificacin nula en ambas condiciones.  Probablemente, son transcritos mal ensamblados que Trinity no pudo concatenar en el mismo grafo De Bruijn en el proceso de ensamblado. Este fenmeno viene a confirmar la ventaja del estadstico E90N50 frente al E100N50 (N50) como refinador de la mediana de la distribucin de longitud de los transcritos.
%
%La informacin de estos transcritos no puede analizarse. Para recuperarla, necesitaramos bien mayor profundidad de secuenciacin, o bien mejores mtodos computacionales de cuantificacin.
%
%En conclusin, los transcritos con menor expresin son ms difciles de procesar en el anlisis de transcriptmica \textit{de novo} y esto causa que Trinity no los ensamble adecuadamente, lo que genera un exceso de transcritos de menor longitud (realmente formaran parte de la misma molcula, pero no pudieron ser ensamblados completamente). A su vez, a consecuencia de ser registrados como \textit{contigs} de menor tamao muy poco abundantes, son difciles de cuantificar, y pueden quedarse fuera del umbral de sensibilidad del software empleado en la cuantificacin (RSEM).
%
%Por todo ello, esperamos que conforme aumenta el nmero de muestras en las que el \textit{contig} de Trinity pudo ser cuantificado por RSEM, aumente su longitud, ya que cuanto ms largo sea, ms fcil es su cuantificacin. El resultado de ese anlisis se muestra en la figura \ref{fig:cortos}.
%
%\subsubsection{La caracterizacin de la dinmica transcripcional permite distinguir grupos de transcritos que correlacionan su expresin}
%
%El resultado de la clusterizacin de los transcritos se refleja en la figura \ref{fig:clusters}. Este tipo de figuras permite visualizar rpidamente picos de expresin de grupos de genes a lo largo de mltiples condiciones y ayuda a entender el proceso global y a formular hiptesis sobre la funcin biolgica de estos grupos. Sin embargo, debido al carcter simplista de este trabajo, solo se analizaron dos condiciones y el grfico obtenido no aporta informacin relevante.
%
%\begin{figure}[!h]
%\begin{center}
%\includegraphics[scale=0.25]{figures/clusters1}% tight frame
%\end{center}
%\caption[Correlacin en la expresin gnica]{Patrones de expresin observados a lo largo de las condiciones. Distinguimos 2 \textit{clusters} de genes agrupados segn el cambio experimentado con el cambio de estado del desarrollo.}
%\label{fig:clusters}
%\end{figure}
%
% Conforme aumentamos el nmero de condiciones exploradas, este tipo de grficos se vuelve ms y ms informativo, siendo la base de la construccin de redes de co-expresin gnica.
%
%
%\nocite{hadley2009}
%\nocite{sonrisa}

%%----------------------------------------------------------------------------------------
%%	BIBLIOGRAPHY
%%----------------------------------------------------------------------------------------
%
\bibliographystyle{unsrt}
%\bibliography{bibliografia}
%%----------------------------------------------------------------------------------------

\end{document}